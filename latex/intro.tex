\intro

%Новизна, актуальность

В последние несколько лет развитие информационных технологий и, в частности, Интернета крайне активно. Многие сервисы и инструменты, которые были актуальны несколько лет назад, сейчас уже считаются устаревшими. Это создает потребность в создании современных инструментов для решения различных задач. Таких инструментов, которые превзойдут своих предшественников по удобству, функциональности или даже принесут что-то качественно новое. Данная работа посвящена раннему этапу разработки одного из таких продуктов. И действительно, как будет далее видно, создание подобного продукта с подобной архитектурой несколько лет назад было бы существенно сложнее, нежели сейчас.

Если мы говорим о новом инструменте, то в первую очередь стоит рассказать о двух вещах: как он называется и какие задачи он решает. Называется он Evergrid. А решаемая задача - это предоставление web-сервиса для проведения вычислительных экспериментов, публикации их результатов вкупе с программной реализацией и данными, предоставление удобного интерфейса для их воспроизведения и модификации. Проект, несомненно, большой - поэтому этот диплом посвящен только его малой части, одной из самых первых и рассматривает решение более узкой задачи нежели создание этой системы целиком.

У проекта есть и другая сторона - вычислительные мощности тоже предоставляются пользователями. Причем необязательно бесплатно. То есть сервис, помимо прочего, является посредником между потребителями и провайдерами ресурсов.

На данный момент аналогичных решений на рынке нет. Как минимум потому, что Evergrid по своей архитектуре не является ни cloud-платформой, ни grid'ом в чистом виде.

В данном дипломе решаются две задачи:

\begin{itemize}
	\item на основе общих принципов Evergrid придумать и описать подходящую для него архитектуру
	\item на основе описанной архитектуры реализовать симулятор для сравнения работы алгоритмов планировщиков
\end{itemize}

В виду ранней стадии разработки проекта описание архитектуры не является доскональным, а скорее задает общую структуру и критичные ограничения. Но, тем не менее, описание достаточно подробно, чтобы на основе него можно было сделать симулятор.

Также рассмотрены существующие симуляторы для grid'ов и обоснована их неэффективность для использования в Evergrid. Показаны сильные стороны выбранного для реализации подхода.

Одной из ключевых особенностей работы является активное использование языка программирования go - именно на нем реализован симулятор.

Реализованный симулятор представляет из себя многоагентную систему. Причем, та его часть, которая реализует среду для выполнения агентов, независима и может использоваться отдельно для решения других задач. Выводом симулятора являются логи в формате JSON, что позволяет проводить произвольный анализ результатов. Код документирован, что облегчит его последующую разработку.