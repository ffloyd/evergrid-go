\intro

В последние несколько лет развитие информационных технологий и в частности Интернета крайне активно. Многие сервисы и инструменты которые были актуальны несколько лет назад сейчас уже считаются устаревшими или неактуальными. Это создает потребность в создании современных инструментов для решения различных задач. Таких инструментов, которые превзойдут своих предшественников по удобству, функциональности или даже принесут что-то качественно новое. Данная работа посвящена раннему этапу разработки одного из таких продуктов. И действительно, как будет далее видно, создание подобного продукта с подобной архитектурой несколько лет назад было бы существенно сложнее нежели сейчас.

Если мы говорим о новом инструменте, то в первую очередь стоит рассказать о двух вещах: как он называется (чтобы было удобнее воспринимать информацию о нем) и какие задачи он решает (чтобы чтение вообще имело смысл). Называется он Evergrid. А решаемая задача - это предоставление web-сервиса для проведения вычислительных экспериментов, публикации их результатов вкупе с программной реализацией и данными, предоставление удобного интерфейса для их воспроизведения и модификации. Проект, несомненно, большой - поэтому этот диплом посвящен только его малой части, одной из самых первых и рассматривает решение более узкой задачи нежели создание этой системы целиком.

Одну и ту же задачу можно решить различными способами. И я говорю даже не об инженерной составляющей, а о сценариях работы. Итак, для пользователя доступны следующие возможности:

\begin{itemize}
	\item Загрузка датасета в систему
	\item Регистрация \textit{процессора} в системе (\textit{процессор} - это специально оформленный docker-контейнер содержащий программную реализацию вычислительного эксперимента)
	\item Запуск выбранного "процессора" на выбранном датасете с определенными условиями (условия - это, например, кол-во денег, которые вы готовы потратить на проведение эксперимента. Ведь не все вычислительные мощности бесплатны)
	\item Публикация результатов
	\item Возможность на основе опубликованных результатов (не только своих) модифицировать эксперимент (изменить алгоритм, использовать другой датасет)
\end{itemize}

Этот краткий список приведен лишь для того, чтобы дать общее представление. На самом деле все немного сложнее и будет подробно описано в соответствующей главе.

Как было упомянуто данный документ посвящен не созданию Evergrid целиком или даже его MVP, а более узкому спектру задач. А именно:

\begin{itemize}
	\item Разработка базовой архитектуры Evergrid
	\item Разработка симулятора этой архитектуры для тестирования работы планировщиков
\end{itemize}

Присутствие первого пункта объясняется тем, что на момент написания работы не существовало описанной в текстовом виде спецификации системы. А для сколь либо продуктивной работы она, очевидно, необходима. Второй пункт - это главная часть работы и основной ее результат. Имея на руках готовую модель можно на более раннем этапе почувствовать недостатки архитектуры и внести в нее изменения до того, как это станет экономически невыгодно. Есть и другие преимущества которые специфичны для созданной реализации в частности, но о них будет рассказано в соответствующих главах.

Поставленные задачи были решены и данный диплом есть результат работы над ними. Достаточно четко для раннего этапа разработки сформулированы основные принципы и архитектура Evergrid, создан симулятор этой архитектуры. Причем созданный симулятор писался не в стиле "написал и забыл", а с пониманием того, что он будет расширяться и модифицироваться другими людьми. Также я держал в голове осознание того, что описанная мной архитектура в будущем, по тем или иным причинам, может измениться и код симулятора должен быть достаточно прост и гибок, чтобы пережить эти изменения.

Последним параграфом введения стоит рассказать о структуре работы. Введение, как видно, написано чтобы дать общее представление о смысле работы. Первые две главы формулируют цели разрабатываемого продукта (Evergrid), его сценарии работы, предлагают архитектуру отвечающую поставленным требованиям и дают рекомендации о выборе технологий для ее реализации. Последующие три главы рассказывают о разработанной среде моделирования. Обосновывают ценность и корректность выбранного пути реализации, показывают как ей пользоваться и демонстрируют результат работы на "тестовых" алгоритмах планировщика. Также внимание уделяется и недостаткам текущей реализации. Последняя глава - это техническая документация созданного инструмента. Она подробно рассматривает его структуру, использованные идиомы и принципы. Помимо этого в ней есть рекомендации о том, как можно расширить и улучшить эту систему. Последняя глава - это главный источник информации для тех, кто продолжит работать над этой частью Evergrid.