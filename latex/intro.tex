\intro

В последние несколько лет развитие информационных технологий и в частности Интернета крайне активно. Многие сервисы и инструменты которые были актуальны несколько лет назад сейчас уже считаются устаревшими или неактуальными. Это создает потребность в создании современных инструментов для решения различных задач. Таких инструментов, которые превзойдут своих предшественников по удобству, функциональности или даже принесут что-то качественно новое. Данная работа посвящена раннему этапу разработки одного из таких продуктов. И действительно, как будет далее видно, создание подобного продукта с подобной архитектурой несколько лет назад было бы существенно сложнее нежели сейчас.

Если мы говорим о новом инструменте, то в первую очередь стоит рассказать о двух вещах: как он называется (чтобы было удобнее воспринимать информацию о нем) и какие задачи он решает (чтобы чтение вообще имело смысл). Называется он Evergrid. А решаемая задача - это предоставление web-сервиса для проведения вычислительных экспериментов, публикации их результатов вкупе с программной реализацией и данными, предоставление удобного интерфейса для их воспроизведения и модификации. Проект, несомненно, большой - поэтому этот диплом посвящен только его малой части, одной из самых первых и рассматривает решение более узкой задачи нежели создание этой системы целиком.

\textit{Краткое описание продукта}

\textit{Сигнал - Действие - Оценка - Результат - Смысл}

Одну и ту же задачу можно решить различными способами. И я говорю даже не об инженерной составляющей, а о сценариях работы. Итак, для пользователя доступны следующие возможности:

\begin{itemize}
	\item Загрузка датасета в систему
	\item Регистрация "процессора" в системе ("процессор" - это специально оформленный docker-контейнер содержащий программную реализацию вычислительного эксперимента)
	\item Запуск выбранного "процессора" на выбранном датасете с определенными условиями (условия - это, например, кол-во денег, которые вы готовы потратить на проведение эксперимента. Ведь не все вычислительные мощности бесплатны)
	\item Публикация результатов
	\item Возможность на основе опубликованных результатов (не только своих) модифицировать эксперимент (изменить алгоритм, использовать другой датасет)
\end{itemize}

Это краткий список приведенный лишь для того, чтобы дать общее представление. На самом деле все немного сложнее и будет подробно описано в соответствующей главе.

\textit{Краткое описание решаемых в дипломе задач}

\textit{Краткое описание результатов}

\textit{Краткое описание структуры текста диплома}