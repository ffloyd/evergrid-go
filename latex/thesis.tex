\documentclass[%
12pt,
master,         % тип документа
natbib,         % использовать пакет natbib для "сжатия" цитирований
subf,           % использовать пакет subcaption для вложенной нумерации рисунков
href,           % использовать пакет hyperref для создания гиперссылок
colorlinks=true % цветные гиперссылки
%,fixint=false  % отключить прямые знаки интегралов
%,times         % шрифт Times как основной
]{disser}

% Original: left - 2.5, right - 1, baseline untouched
\renewcommand{\baselinestretch}{1.5}
\usepackage[
  a4paper, mag=1000,
  left=3cm, right=2cm, top=2cm, bottom=2cm, headsep=0.7cm, footskip=1cm
]{geometry}
\usepackage[T2A]{fontenc}
\usepackage[utf8]{inputenc}
\usepackage[english,russian]{babel}
%\usepackage{tabularx,longtable}
\ifpdf\usepackage{epstopdf}\fi

% Номера страниц снизу и по центру
%\pagestyle{footcenter}
%\chapterpagestyle{footcenter}

% Точка с запятой в качестве разделителя между номерами цитирований
%\setcitestyle{semicolon}

% Использовать полужирное начертание для векторов
\let\vec=\mathbf

% Включать подсекции в оглавление
\setcounter{tocdepth}{2}

\graphicspath{{fig/}}

\begin{document}

% Переопределение стандартных заголовков
%\def\contentsname{Содержание}
%\def\conclusionname{Выводы}
%\def\bibname{Литература}

\institution{Министерство образования и науки Российской Федерации\\
			Московский физико-технический институт {\rm(государственный университет)}\\
			Факультет управления и прикладной математики\\
			Кафедра информатики}

% Имя лица, допускающего к защите (зав. кафедрой)
\apname{Петров Игорь Борисович}

\title{ДИССЕРТАЦИЯ\\[-14pt]на соискание ученой степени\\МАГИСТРА}

\topic{Разработка модели среды распределенных вычислений Evergrid для сравнения алгоритмов управления потоками задач и данных}

% Автор
\author       {Колеснев Роман Владимирович} % ФИО
\group        {073а} % Группа
\coursenum    {511656} % Номер направления
\course       {Математические и информационные технологии}
\masterprognum{????} % Номер магистерской программы
\masterprog   {Название программы}

% Научный руководитель
\sa      {Устюжанин Андрей Евгеньевич}
\sastatus{к.~ф.-м.~н., доцент}

% Рецензент
\rev      {Рыков Владимир Васильевич}
\revstatus{к.~л. ~н., доцент}
% Второй рецензент
%\revsnd      {ФИО рецензента}
%\revsndstatus{д.~т.~н., ст.~н.~с.}

% Консультант
%\con{ФИО консультанта}
%\conspec{вопросам\\охраны труда}
%\constatus{к.~т.~н., доц.}
% Второй консультант
%\consnd{ФИО консультанта}
%\consndspec{экономическим\\вопросам}
%\consndstatus{к.~э.~н., доц.}

% Город и год
\city{Москва}
\date{\number2016}

\maketitle

% Содержание
\tableofcontents
% Введение
\intro

В последние несколько лет развитие информационных технологий и в частности Интернета крайне активно. Многие сервисы и инструменты которые были актуальны несколько лет назад сейчас уже считаются устаревшими или неактуальными. Это создает потребность в создании современных инструментов для решения различных задач. Таких инструментов, которые превзойдут своих предшественников по удобству, функциональности или даже принесут что-то качественно новое. Данная работа посвящена раннему этапу разработки одного из таких продуктов. И действительно, как будет далее видно, создание подобного продукта с подобной архитектурой несколько лет назад было бы существенно сложнее нежели сейчас.

Если мы говорим о новом инструменте, то в первую очередь стоит рассказать о двух вещах: как он называется (чтобы было удобнее воспринимать информацию о нем) и какие задачи он решает (чтобы чтение вообще имело смысл). Называется он Evergrid. А решаемая задача - это предоставление web-сервиса для проведения вычислительных экспериментов, публикации их результатов вкупе с программной реализацией и данными, предоставление удобного интерфейса для их воспроизведения и модификации. Проект, несомненно, большой - поэтому этот диплом посвящен только его малой части, одной из самых первых и рассматривает решение более узкой задачи нежели создание этой системы целиком.

Одну и ту же задачу можно решить различными способами. И я говорю даже не об инженерной составляющей, а о сценариях работы. Итак, для пользователя доступны следующие возможности:

\begin{itemize}
	\item Загрузка датасета в систему
	\item Регистрация \textit{процессора} в системе (\textit{процессор} - это специально оформленный docker-контейнер содержащий программную реализацию вычислительного эксперимента)
	\item Запуск выбранного "процессора" на выбранном датасете с определенными условиями (условия - это, например, кол-во денег, которые вы готовы потратить на проведение эксперимента. Ведь не все вычислительные мощности бесплатны)
	\item Публикация результатов
	\item Возможность на основе опубликованных результатов (не только своих) модифицировать эксперимент (изменить алгоритм, использовать другой датасет)
\end{itemize}

Этот краткий список приведен лишь для того, чтобы дать общее представление. На самом деле все немного сложнее и будет подробно описано в соответствующей главе.

Как было упомянуто данный документ посвящен не созданию Evergrid целиком или даже его MVP, а более узкому спектру задач. А именно:

\begin{itemize}
	\item Разработка базовой архитектуры Evergrid
	\item Разработка симулятора этой архитектуры для тестирования работы планировщиков
\end{itemize}

Присутствие первого пункта объясняется тем, что на момент написания работы не существовало описанной в текстовом виде спецификации системы. А для сколь либо продуктивной работы она, очевидно, необходима. Второй пункт - это главная часть работы и основной ее результат. Имея на руках готовую модель можно на более раннем этапе почувствовать недостатки архитектуры и внести в нее изменения до того, как это станет экономически невыгодно. Есть и другие преимущества которые специфичны для созданной реализации в частности, но о них будет рассказано в соответствующих главах.

Поставленные задачи были решены и данный диплом есть результат работы над ними. Достаточно четко для раннего этапа разработки сформулированы основные принципы и архитектура Evergrid, создан симулятор этой архитектуры. Причем созданный симулятор писался не в стиле "написал и забыл", а с пониманием того, что он будет расширяться и модифицироваться другими людьми. Также я держал в голове осознание того, что описанная мной архитектура в будущем, по тем или иным причинам, может измениться и код симулятора должен быть достаточно прост и гибок, чтобы пережить эти изменения.

Последним параграфом введения стоит рассказать о структуре работы. Введение, как видно, написано чтобы дать общее представление о смысле работы. Первые две главы формулируют цели разрабатываемого продукта (Evergrid), его сценарии работы, предлагают архитектуру отвечающую поставленным требованиям и дают рекомендации о выборе технологий для ее реализации. Последующие три главы рассказывают о разработанной среде моделирования. Обосновывают ценность и корректность выбранного пути реализации, показывают как ей пользоваться и демонстрируют результат работы на "тестовых" алгоритмах планировщика. Также внимание уделяется и недостаткам текущей реализации. Последняя глава - это техническая документация созданного инструмента. Она подробно рассматривает его структуру, использованные идиомы и принципы. Помимо этого в ней есть рекомендации о том, как можно расширить и улучшить эту систему. Последняя глава - это главный источник информации для тех, кто продолжит работать над этой частью Evergrid.
% Глава 1 - Постановка глобальной задачи
\chapter{Постановка глобальной задачи}

\textit{Уточнить, что подразумевается под словами "глобальная задача"}

\textit{Сигнал - Действие - Оценка - Результат - Смысл}

\textit{О распространенности экспериментов над данными}

\textit{О важности эффективной коммуникации}

\textit{О сложностях с попытками воспроизвести тот или иной эксперимент}

\textit{Об открытом доступе к результатам эксперимента}

\textit{О доступности модификации эксперимента}

\textit{О необходимости большого количества вычислительных мощностей. Второй тип пользователей.}

\textit{Evergrid - платформа для решения этих проблем.}

\textit{Суммарный список возможностей}
% Глава 2 - Постановка решаемых в дипломе задач
\chapter{Постановка решаемых в дипломе задач}

Пусть список возможностей из предыдущей главы получился весьма лаконичным его реализация - это огромная работа. Напомню, что данный диплом представляет собой лишь один из самых первых этапов - это имеет прямое влияние на те задачи которые непосредственно решались и на то как они решались.

Краткий список того, над чем велась работа выглядит так:

\begin{itemize}
	\item Четко сформулировать базовые требования к системе Evergrid
	\item Разработать описание архитектуры удовлетворяющее этим требованиям
	\item Для данной архитектуры реализовать среду моделирования для исследования эффективности планировщиков выполнения задач и распределения данных
\end{itemize}

Базовые требования были описаны и обоснованны в предыдущей главе.

Наличие описания архитектуры является необходимым условием для создания среды моделирования, которая, в свою очередь, является темой диплома. Мало того, что оно должно быть - это описание с одной стороны должно быть достаточно подробным, с другой - максимально общим, чтобы не "замораживать" спецификацию тех аспектов системы, детали реализации которых несущественны для данной работы. Тем не менее даже для таких аспектов будет не лишним дать некоторую обоснованную рекомендацию - она может послужить удачной точкой для принятия финального решения в будущем. Именно с таким пониманием вопроса велась работа об описании архитектуры.

Касательно описания архитектуры отдельно хочу отметить решения о выборе языков и технологий для реализации. Конечно, все они имеют по большей части рекомендательный характер сопровожденный развернутым комментарием "почему подобный выбор хорош". Единственное исключение - это использование языка go для написания внутренних сервисов приложения и его распределенной части. Причины этого выбора и сравнение с альтернативными решениями будут рассмотрены отдельно и более подробно. Забегая вперед скажу, что среда моделирования написана на go с расчетом на то, что и определенный набор компонентов тоже будет написан на go. Именно из-за этого этому техническому выбору уделяется особое внимание.

Теперь к основному: среда моделирования. Я не буду сейчас вдаваться в подробности реализации и принципы ее работы. По сути, постановка задачи здесь получается достаточно вольной: необходимо иметь возможность тестировать и сравнивать работу различных алгоритмов планировщиков. Конкретные требования к среде моделирования, ее особенности и конкурентные преимущества сформировались в процессе работы над дипломом. Однако есть одно свойство, о котором можно смело говорить до того, как будет описана архитектура Evergrid: полученный результат должен быть хорошо продокументирован и удобен для модификации. Это требование естественным образом происходит из факта раннего этапа разработки самого Evergrid.

Помимо непосредственно реализации среды моделирования нужны две-три несложных реализации алгоритмов планировщика. Причем они должны быть различными по своей сути и приоритетам, выдавать предсказуемый результат. Из-за своей тривиальности они могут послужить хорошей исходной точкой для сравнения - если разработанный боевой алгоритм работает хуже тривиального - значит он не такой уж и "боевой". 
% Глава 3 - Архитектура продукта
\chapter{Спроектированная архитектура}

\section{Основные ограничения}

Проектирование хорошей архитектуры основывается на:

\begin{itemize}
	\item достаточно подробной постановке задачи
	\item выявлении ограничений
	\item использовании актуальных технологий как составных блоков
	\item максимальной простоте реализации не в ущерб необходимому уровню качества
\end{itemize}

Постановка задачи у нас есть, последние два пункта постараемся соблюдать. Осталось понять, какие у нас ограничения. Под ограничениями понимаются технические аспекты, которые напрямую следуют из постановки задачи и ограничивают нас в свободе выбора тех или иных технологий, тех или иных подходов. В этом разделе я приведу те ограничения, которые считаются неочевидными или заслуживающими отдельного упоминания.

Первое из них - связанное с безопасностью. Мы используем вычислительные ресурсы, предоставляемые третьими лицами. И они имеют неограниченный (root) доступ к ним. Поскольку процесс предоставления этих ресурсов подразумевается достаточно свободным, будет правильно представить злоумышленника на месте поставщика мощностей. Какие потенциальные угрозы он может предоставлять?

\begin{itemize}
	\item захват не принадлежащих ему ресурсов
	\item нарушение работы кластера (например, вмешательство в работу планировщика)
	\item нарушение корректной работы предоставленного вычислительного ресурса (фальсификация результатов, намеренное замедление скорости вычислений и т. п.)
\end{itemize}

Первые две угрозы нивелируются достаточно просто: \textit{на арендуемых ресурсах не должен выполняться код, связанный с управлением кластером. Только выполнение задач и отправка результатов.}

Третья - наиболее сложная. Но риски можно свести к минимуму, если \textit{вообще не запускать на арендуемых ресурсах компоненты системы}. Естественной реализацией этого принципа является удаленное управление по SSH. Тогда злоумышленник будет знать минимум о текущем состоянии системы. Прочие методы борьбы с этой угрозой уже выходят за рамки этой работы.

Следующее ограничение происходит из того, как мы будем выполнять задачи. Очевидно, что нам нужна виртуализация - иначе наш кластер быстро превратят в ботнет. Нам нужна универсальность - хочется покрыть как можно больше сценариев использования. Также нам нужна скорость - следовательно, нам нужна легковесная виртуализация. Третий критерий - технология должна быть простой в использовании. В идеале - проведение эксперимента на своей локальной машине не должно отличаться от выполнения его на наших мощностях. Сложив все эти требования воедино, мы получим ответ - Docker. И это решение имеет много преимуществ: Docker отлично отвечает требованиям к изоляции, удобству воспроизводимости и модификации. А его инструментарий органично вписывается в архитектуру. От предоставляемых мощностей нам в итоге требуется:

\begin{itemize}
	\item свободное место на диске
	\item доступ по ssh
	\item корректно работающий docker
	\item возможность мониторинга нагрузки (чтобы регистрировать не относящуюся к нашему сервису нагрузку)
\end{itemize}

Даже неограниченный доступ в Интернет не является жестким требованием - мы можем собрать контейнер на наших машинах и готовый образ передать по ssh.

Возникает вопрос о том, как пользователь должен предоставлять контейнер со своим алгоритмом. Если следовать идеям доступности и простоты модификации, то наиболее очевидное решение - Github. Наиболее органично будет работать с специально оформленными github-репозиториями, которые содержат все необходимое для сборки контейнера.

Использование Docker на первый взгляд накладывает ограничение вида ,,одна задача = одна машина'', но это можно обойти - ведь мы можем активизировать несколько машин и разрешить контейнерам общаться между собой. Тем не менее, в рамках данной работы акцент сделан именно на подходе "одна задача = одна машина". Несмотря на это, архитектура должна быть пригодна для обоих вариантов.

Если уж рассмотрели то, как загружать реализации алгоритмов (далее будем называть их контейнерами, раз имеем в виду Docker) - то стоит сказать пару слов о загрузке датасетов. Никаких особых ограничений здесь незаметно. Все, что нужно - это иметь системе возможность закачать собранный контейнер на свои машины. Путей достижения этого много, и в данной работе они не будут рассматриваться (за исключением перемещения датасетов внутри самой системы).

И последнее ограничение, о котором я хочу сказать, - это CAP-теорема. Мы имеем дело с распределенной в глобальной сети системой, а в таком случае нельзя забывать о CAP-теореме. Нарушение связи между элементами системы не должно приводить к ее некорректной работе.

\pagebreak

\section{Предлагаемая архитектура}

\begin{figure}
	\centering
	\includegraphics[width=\textwidth]{fig/architecture}
	\caption{Диаграмма архитектуры}\label{fig:architecture}
\end{figure}

Теперь можно рассмотреть конкретную реализацию, удовлетворяющую описанным выше принципам. Диаграмма предлагаемой архитектуры показана на рисунке \ref{fig:architecture}. Далее мы рассмотрим в отдельности каждый из ее слоев и компонентов.

Для начала объясню общую структуру. Здесь видно разделение на три слоя плюс пользователи системы. User 1 -- User X - это пользователи. Каждый слой состоит из компонентов. Распределение компонентов по машинам специфично для каждого слоя. Для слоя Interface оно явно не ограничивается предложенной архитектурой. В зависимости от нагрузки и прочих факторов все эти компоненты могут быть как на одной машине, так и разнесены по нескольким. Для слоя Control истинно правило "одна машина - один Control Unit". Для слоя Execution все просто - один Worker равен одной машине, предоставленной для вычислений.

Направление стрелок - это направление запросов между компонентами. В случае долговременного соединения - стрелка направлена от инициатора.

Первые два слоя - это “наши” машины, которым мы доверяем, и только у нас есть к ним доступ.

Третий слой - машины, предоставленные извне, мы им “не доверяем”.

Слева показаны приоритеты в терминах CAP-теоремы для каждого слоя.

Справа показано, что первые два слоя используют Hashicorp Consul для service discovery и leader election.

\subsection{Слой Execution}

На этом слое находятся машины, предоставленные пользователями системы. Они частично конфигурируются нами и не содержат никакого кода, который бы управлял слоями выше (из соображений безопасности). Про специфику работы с этими ресурсами и требования к ним было написано выше.

Все запросы приходят со слоя Control. То есть машинам даже необязательно знать IP других компонентов системы. Отдельный случай - если машины не имеют ''белого'' IP: в этом случае в их обязанность входит самостоятельно поддерживать соединение со слоем Control, и нам все же придется ставить на Worker'ы какое-нибудь приложение, которое будет обеспечивать соединение со слоем Core.

\subsection{Слой Control}

Состоит только из Control Unit’ов. Каждый Control Unit - это отдельная машина. Каждый Worker из слоя ниже принадлежит только одному Сontrol Unit’y.

Control Unit состоит из нескольких логических компонент и представляет из себя монолитную программу.

Все Control Unit с этого слоя представляют из себя распределенную систему. То есть запросы с верхнего слоя Interface по своей сути направлены не к конкретному Control Unit'у, а ко всей системе целиком.

Смысл подобного разбиения - это большая стабильность системы. Т. е. эффективным подходом будет, например, расположить по одному Control Unit на географическую зону.

Смысл всей этой системы в следующем:

\begin{itemize}
	\item Получать задания со слоя выше (от Сore)
	\item Раскидывать датасеты по Worker’ам
	\item Запускать вычисления на Worker’ах
	\item Отправлять результаты вычислений на Worker’ах на слой выше (в Core).
\end{itemize}

Теперь конкретные примеры запросов, которые могут приходить в эту систему:

\begin{itemize}
	\item Загрузи датасет
	\item Загрузи и собери этот docker-контейнер
	\item Обнови этот докер контейнер
	\item Обнови этот датасет
	\item Запусти этот контейнер с этим датасетом
\end{itemize}

Из подобных задач формируются локальные очереди (Local Queue). В какую локальную очередь и на какой Worker отправить задачу, определяет “распределенный” Scheduler. Также состояние, корректность и доступность Worker'ов, прочие глобальные характеристики системы отслеживаются Monitor'ом. Monitor может давать информацию о всех доступных Worker'ах системы. Monitor - основной инструмент Scheduler'а для получения текущего состояния системы. Executor отвечает за "опустошение" очереди и выполнение задач на Worker'ах.

Результат выполнения работы отправляется в Core, который, в свою очередь, решает, где хранить этот результат.

\subsection{Слой Interface}

Распределение по машинам в этом слое диктуется лишь нагрузкой. При малом количестве пользователей - можно все три компонента держать на одной машине. Теперь отдельно про каждый компонент.

\begin{description}
  \item[WEB] Веб-приложение, написанное на любом адекватном web-фреймворке. Именно через него происходит постановка задач, просмотр результатов и прочее. Является единой точкой управления системой как для рядовых пользователей, так и для администраторов. Важно понимать, что здесь используются не только HTML + JS, но и реализована вся бизнес-логика интерфейса.
  \item[DB] Основная база данных. В ней хранится все относящееся к бизнес-логике приложения. По своему содержанию должна быть независима от нижестоящих слоев. Т. е. если конфигурация всего того, что есть ниже, станет иной - данные не должны стать некорректными.
  \item[Core] WEB не общается напрямую со слоем Сontrol. Он работает с Core через API, а оно в свою очередь контролирует слой ниже. Т. е. Core - это микросервис взаимодействия со слоем Control. Это важное разделение, поэтому подчеркну: WEB - это бизнес логика, Core - микросервис, который контролирует общение с распределенной вычислительной системой. Core не проверяет никаких параметров, связанных с бизнес-логикой: кто поставил задачу, сколько у него денег на счету и прочее. 
\end{description}

\section{Выбор технологий для реализаций компонентов}

Теперь, когда есть представление о структуре, надо понять, какие есть эффективные способы реализации. Если не оговорено иного, предложенные способы несут рекомендательный характер.

\subsection{WEB}

Посколько Evergrid является по своей сути стартапом, то важным аспектом является скорость разработки. То есть надо уметь быстро создать прототип и иметь возможно быстро вносить изменения. На Java и ASP.NET подобное получается плохо, особенно у небольших команд. Наиболее успешно используемые технологии - это Ruby on Rails\footnote{\url{http://rubyonrails.org/}} , Node.js\footnote{\url{https://nodejs.org/en/}} фреймворки\footnote{Например: \url{http://expressjs.com/}} , Django\footnote{\url{https://www.djangoproject.com/}} и прочие python-фреймворки, PHP\footnote{\url{http://php.net/}}.

Есть также менее распространенные решения, но тоже заслуживающие внимания: Play Framework\footnote{\url{https://www.playframework.com/}} , Phoenix Framework\footnote{\url{http://www.phoenixframework.org/}}.

Веб-фреймворки можно разделить на два класса:

\begin{description}
  \item[С "синхронной" архитектурой] -- когда запускается ровно N обработчиков запросов (в отдельных процессах или тредах). То есть система может одновременно обрабатывать не более чем N запросов. Выполнение запросов изолировано друг от друга. В данных ограничениях удобно писать код, но страдает масштабируемость и не стоит использовать продолжительные по времени подключения (websocket'ы, например). Синхронным такой подход называется т. к. в случае одного обработчика мы блокируем выполнение следующего запроса до завершения предыдущего.
  \item[С "асинхронной" архитектурой] -- в этом подходе для каждого входящего запроса на лету создается свой поток обработки. Т. е. мы, в отличие от предыдущего подхода, не вынуждены ждать завершения обработки предыдущего запроса для начала обработки нового. Для такой архитектуры сложнее писать код, но обычно решения на ней обладают лучшей производительностью и более широким спектром возможностей.
  \item
\end{description}

Это описание, достаточное для общего понимания. Еще один важный параметр - это используемый язык программирования. От него зависит как скорость продукта, так и, отчасти, легкость его сопровождения.

Сначала рассмотрим технологии, использование которых может оказаться неэффективным:

\begin{description}
	\item[Java/ASP.NET] Плохо годятся для небольших команд. ASP.NET плох своим акцентам на Windows-сервера. Java, на фоне прочих языков, слишком "многословна", и быстро реализовать на ней прототип довольно тяжело.
	\item[Node.js] Несмотря на асинхронность архитектуры, javasript работает в одном потоке. Но главный его недостаток - это дизайн языка. У JS крайне слабая система типов, запутанная спецификация и много неоправданно неочевидных моментов.
	\item[PHP] Очень популярен в вебе, до сих пор развивается, множество фреймворков. Но дизайн языка хоть и лучше, чем у JS, все равно уступает ruby и python. Другой его недостаток - слабость как языка общего назначения - выражается в том, что на нем неудобно писать что-то кроме непосредственно генерации ответов на запросы к веб-серверу. А подобное может понадобиться - например, для организации выполнения внешних очередей задач или удобного использования websocket'ов. Кратко говоря, PHP неоправданно накладывает слишком много ограничений.
\end{description}

Теперь рассмотрим список технологий, хорошо подходящих этому проекту:

\begin{description}
	\item[Python] Python активно используется как для веб-разработки, так и для научных целей. Веб-фреймворки на нем довольно качественные, и есть большое количество решений для типовых задач (например, регистрация пользователей и пр.). Поскольку Python скриптовый язык с GIL, большинство решений на нем реализуют синхронную архитектуру. Это не настолько большой недостаток, как может показаться, но, если высокая производительность или использование websocket'ов является критичным, - python может оказаться не лучшим выбором.
	\item[Ruby on Rails] В основном ситуация, схожая с python за исключением следующих различий: редко используется в научной среде; язык более гибкий, но и более сложный, чем python; большее количество библиотек с готовыми решениями для веб-разработки. Отдельным преимуществом является и то, что Rails однозначный лидер среди Ruby-фреймворков, а это означает, что почти любая развитая библиотека умеет корректно с ним работать. В случае с python все более разобщено. То есть, если есть возможность эффективно использовать Rails - то это является удачным решением в рамках синхронной архитектуры.
	\item[Phoenix Framework (Elixir)] Если важно использование асинхронной архитектуры, то стоит вспомнить об Erlang. Продукты, написанные на Erlang, в виду особенностей BEAM (Bogdan/Björn’s Erlang Abstract Machine) и принципов OTP (Open Telecom Platform) отличаются высокой стабильностью и производительностью. Elixir - это молодой язык для BEAM, который более удобен в использовании, имеет больше возможностей, чем Erlang, и может без ограничений использовать его библиотеки. На этом языке реализован Phoenix Framework. Что язык, что сам фреймворк - еще молодые, используются в production, но пока еще не получили широкого распространения. Сам фреймворк удобен в использовании. Данный вариант хорош, если хочется получить преимущества Erlang/OTP-экосистемы: высокая стабильность, высокая производительность на IO задачах, дешевый и удобный параллелизм благодаря green threads и акторной модели, преимущества подходов из функциональной парадигмы программирования.
	\item[Play Framework (Scala)] Еще одно возможное решение, если нужна асинхронность и производительность. Scala - преимущественно функциональный язык программирования поверх JVM, имеющий возможности интероперабельности с Java. Тоже использует акторную модель (Akka). Более зрелое решение, чем phoenix, и более широко распространено на данный момент. Из недостатков можно отметить то, что в Erlang более развитые средства для мониторинга и более стабильная реализация акторной модели и супервизоров. Также Scala является более сложным в изучении и использовании языком, нежели Elixir/Erlang. Из преимуществ можно отметить, что мы получаем доступ к внушительному парку Scala и Java библиотек.
\end{description}


\subsection{Core и Control Unit}

Эти два компонента формируют внутреннюю инфраструктуру сервиса. В данном случае корректным будет следующий список требований:

\begin{itemize}
	\item высокая производительность (система должна адекватно вести себя под высокими нагрузками)
	\item выбранное решение должно использоваться для схожих известных продуктов
	\item иметь развитые библиотеки для сетевого взаимодействия
	\item чем проще будет инициализировать новые сервера - тем лучше
	\item язык должен быть популярен и иметь активное сообщество
	\item простота языка будет плюсом
\end{itemize}

Требование высокой производительности ставит под сомнение использование интерпретируемых языков. Из "быстрых" языков в первую очередь приходят на ум C/C++ и go. ASP.NET плохо дружит с Linux, а Java/Scala довольно прожорливы в плане памяти и не являются простыми в использовании языками. С++ - тоже довольно сложный в использовании язык. Если же внимательнее посмотреть на go, то становится видно, что он является хорошим решением:

\begin{itemize}
	\item на нем написаны продукты hashicorp (nomad, consul), на нем написан docker
	\item скорость выполнения сравнима с C/C++
	\item прост в изучении: опытный программист может освоить все основные возможности языка за пару вечеров
	\item популярен для написания микросервисов
	\item встроенная в язык модель параллелизма CSP
	\item активное сообщество и большое количество готовых библиотек
	\item скомпилированное приложение для своей работы не требует установки самого go или каких-либо специфических пакетов
\end{itemize}

Поэтому для Core и Control Unit go является подходящим выбором. Причем данный выбор имеет жесткий характер, так как симулятор тоже написан на go, и часть его преимуществ основывается на том, что для реализации инфраструктуры тоже будет использоваться этот язык.

\subsection{DB}

На самом деле на данном этапе сложно предсказать все те требования, которые могут возникнуть для основного хранилища. Возможно, даже одного продукта не хватит, и будет использоваться некая комбинация (например: часто можно встретить связки Postgres + Redis).

Но, в качестве решения по умолчанию, стоит рассматривать именно Postgres. На данный момент это наиболее универсальное и активно используемое решение среди open source баз данных.

\section{Ограничения, связанные с CAP-теоремой}

Поскольку мы имеем дело с распределенной системой, то стоит явно указать приоритеты в терминах CAP-теоремы. Причем эти приоритеты отдельные для каждого слоя (именно из-за этого вообще появилось разделение на слои).

Также первый и второй слой используют consul - он предоставляет service discovery и leader election интерфейсы, к тому же он создан для решения этих задач как раз-таки в распределенных системах.


\begin{description}
  \item[Слой Execution] Для этого слоя подходит конфигурация [C]AP. Если машина потеряла связь с системой -- перераспределяем нагрузку. Здесь важно в любой момент времени максимально использовать доступные ресурсы.
  \item[Слой Control] При критичной сегментации сети в слое, Control Unit'ы перестают принимать запросы, но продолжают выполнение локальных очередей. Примерно так выглядит жертва availability во имя consistency и partition tolerance. Стоит сказать, что это довольно грубый подход. Но начать разработку проще именно с него. Как второй шаг может подойти weak consistency подход ("согласованность в итоге").
  \item[Слой Interface] На этом слое крайне важна согласованность данных, а без доступности он не имеет смысла. Поэтому жертвуем partition tolerance.
\end{description}

\section{Требования к симулятору}

Имея описание архитектуры, можно сформулировать список требований к симулятору.

Первое из них связано с симуляцией сети. Для симуляции данной архитектуры нам не важен пинг между ее компонентами. Нам важна только скорость передачи данных между ними и сам факт наличия связи. Отдельный случай - это черезмерно большой пинг - это может повлиять на порядок запросов и подобное. Но по факту нам нужно симулировать не пинг, а факт поздней доставки сообщений. В итоге имеем следующие требования:

\begin{itemize}
	\item симуляция видимости между машинами
	\item симуляция скорости передачи данных
	\item симуляция разрыва соединения
	\item симуляция поздно пришедших сообщений
\end{itemize}

Второе ограничение связанно с тем, что мы должны симулировать время. Т. к. нам важно понимать когда выполнялась задача и сколько времени она выполнялась. Наиболее гибким простым решением является "дискретная" природа симулируемого времени. То есть мы используем дискретный таймер, и каждый агент системы делает ту работу, которую теоретически может успеть за эту дискретную единицу времени.

И третье ограничение - симулятор должен быть достаточно гибким, чтобы при изменении принципов работы компонент можно было легко внести изменения в модель.

Всякий симулятор преследует цель проверки гипотез. В данном случае симулятор должен помогать в исследовании:

\begin{itemize}
	\item эффективности алгоритмов планировщиков
	\item принципов взаимодействия компонентов системы
\end{itemize}
% Глава 4 - Требования к среде моделирования
\chapter{Сравнение с существующими решениями}

В рамках работы была рассмотрена возможность использования или модификации существующего решения вместо написания своей реализации. Для корректного сравнения стоит сразу перечислить преимущества написания своей реализации на go. Если взять во внимание тот факт, что подобных приложений на go найдено не было, то преимущества таковы:

\begin{itemize}
	\item возможность использовать один код планировщика как для симуляции, так и для реального окружения - это дает дополнительные гарантии корректности его реализации
	\item удобная возможность записи реальных сценариев работы и последующей симуляции их с различными вариантами планировщиков
	\item go - простой в использовании язык и имеет встроенную модель параллелизма CSP, это облегчает разработку и модификацию кода симуляции
	\item go - быстрый язык, скорость его работы сравнима с C/C++
	\item в go встроен автоматический детектор состояний гонки
	\item go популярен для микросервисов, но на данный момент нет ни одного симулятора для go
\end{itemize}

Следует понимать, что использование или модификация готового решения должны предоставлять преимущества, перевешивающие этот список.

\section{NetLogo}

NetLogo\cite{tisue2004netlogo} - очень известное решение для моделирования и исследования работы многоагентных систем. Его преимущество состоит в развитых инструментах визуализации.

Но для решения конкретной задачи NetLogo подходит плохо, т. к.:

\begin{itemize}
	\item NetLogo использует свой язык программирования. Это язык узкого назначения и писать на нем менее удобно, чем на популярных языках общего назначения
	\item мы симулируем не только scheduler, но и прочие компоненты системы. Нам важно, чтобы модель общения между ними была приближена к реальной. Это дает возможности для поиска состояний гонки в рамках архитектуры в целом. Реализация честного общения всех компонентов довольно сложна на NetLogo и потенциально сложнее написания своей реализации на go.
	\item у него низкая скорость работы
\end{itemize}

\section{Узкоспециализированные симуляторы: SimGrid, GridSim, ALEA 2}

Среди более узкоспециализированных решений было найдено три кандидата: SimGrid\cite{casanova2001simgrid}, GridSim\cite{GridSim} и ALEA 2\cite{ALEA}.

Эти продукты реализованы на Java (GridSim, ALEA 2) и C (SimGrid). У всех трех есть один критический недостаток: без модификации исходного кода невозможно реализовать поставленную задачу - архитектура проекта слишком специфична. А сама задача их модификации получается сложнее, чем написание своего решения. Также усложняется и сопровождение получившегося симулятора - вносить правки тяжелее по двум причинами: Java и C менее удобны чем go, больший шанс неожиданных ошибок в виду возможных конфликтов с изначальной архитектурой симулятора.

Еще одна особенность всех трех симуляторов - они сверхсконцентрированы на анализе планировщика, в то время как одной из наших задач является и анализ архитектуры в целом.

Также есть проблемы специфичные для каждого из решений:

\subsection{GridSim}

Дата последнего обновления - 2010й год. Видимо, проект более не поддерживается, и при возникновении незадокументированных проблем трудно будет связаться с его разработчиком.

Также при использовании GridSim будет проблематично симулировать "динамическое" окружение - разрывы сети, меняющийся список воркеров и прочее.

Итого: суммарная сложность модификации и поддержки решения на GridSim потенциально выше сложности написания и поддержки своей реализации на go. При этом мы еще жертвуем вышеописанными преимуществами своей реализации.

\subsection{ALEA 2}

ALEA 2 является модификацией GridSim, и ее использование сопряжено с теми же проблемами. В отличие от GridSim, она еще поддерживается разработчиками.

\subsection{SimGrid}

Наиболее активно поддерживаемый продукт, но более узкоспециализирован, чем два других. Количество модификаций, которые необходимо будет внести, существенно больше - что  усложняет его адаптацию.

К тому же, из соображений скорости он написан на С, а модифицировать код на С обычно сложнее, чем модифицировать код на Java.
% Глава 5 - Анализ возможных реализаций
\chapter{Реализованная в симуляторе модель}

В этой главе описана реализованная в симуляторе модель. Это даст представление о том, как проходит процесс симуляции и какова степень подробности и точности симулятора.

\section{Многоагентное окружение с дискретным временем}

Основу симулятора представляет из себя многоагентая среда. Она состоит из агентов и ядра, которые общаются друг с другом посредством сообщений.

Ядро системы обеспечивает синхронизацию работы агентов и управляет временем. Симуляция не является непрерывной и разбита на дискретные единицы времени (ticks). Каждый tick соответствует одной минуте. Подобный выбор масштаба обуславливается тем, что предполагается достаточная скорость работы планировщика, чтобы успеть обработать запросы пришедшие в рамках одного tick'а за минуту. А исходя из того, что предполагаются довольно большие объемы датасетов, погрешность времени порядка минуты на передачу данных является допустимой.

\begin{figure}
	\centering
	\includegraphics{fig/agent_workflow}
	\caption{Цикл работы агента}\label{fig:agent_workflow}
\end{figure}

В процессе каждого тика агент меняет свое состояние. Данный процесс цикличен. Возможные состояния агента и переходы между ними показаны на рисунке \ref{fig:agent_workflow}. Прямоугольники - это состояния. Второй тип элементов - это синхронизации. Синхронизация означает, что невозможно пройти по данному ребру, пока все агенты не будут готовы.

Описание состояний:

\begin{description}
	\item[Done] Означает, что предыдущий тик завершен успешно. Является исходным состоянием агента.
	\item[Ready] Означает, что агент готов начать работу в рамках нового тика.
	\item[Working] Означает, что агент в данный момент совершает работу.
	\item[Idle] Означает, что агент в данный момент бездействует и выполнил всю делегированную ему на данный момент работу.
\end{description}

Описание синхронизаций:

\begin{description}
	\item[Start Work] Дает гарантию, что перед началом работы все агенты достигли состояния Ready
	\item[Finish Work] Дает гарантию, что перед завершением тика все агенты находятся в состоянии Idle.
\end{description}

Важным аспектом является то, что признак завершения тика - это статус Idle у всех агентов системы. Это нужно помнить и не позволять системе попадать в это состояние, когда не вся работа сделана.

В состоянии Idle агент ждет одного из двух событий: новое сообщение от другого агента либо сигнал Finish Work. В последнем случае состояние меняется на Done. Эта смена состояния единственная не контролируется агентом напрямую.

Помимо основного статуса каждый агент имеет булев маркер stopFlag. Если он равен true, значит агент не запланировал никакой работы на будущие тики, и, если симуляция завершится сейчас, то это будет корректный исход.

Соответственно, симуляция завершается, когда у всех агентов stopFlag маркер равен true.

\section{Сетевое окружение}

В терминах описанной многоагентной среды симуляция сетевого окружения является заботой самих агентов. Причем это несложно реализуется: если агент знает свое место и места прочих агентов внутри сети, то он сам может накладывать ограничения на общения с этими агентами. То есть симуляция сети сводится к предоставлению конкретным агентам информации о конфигурации сети.

Структура сети в текущей версии симулятора представляет из себя набор из сегментов. Каждый сегмент содержит несколько машин. К одной машине могут быть привязаны один и более агентов.

Скорость передачи данных внутри сегмента и между сегментами может быть различной и является частью его описания.

\section{Параллелизм и коммуникация}

Агенты должны исполняться параллельно друг другу и ядру. Конечно, они могут блокировать друг друга - это соответствует ситуации, когда сервис не может ответить на новый запрос, пока не завершена обработка предыдущего. Но в рамках независимых участков кода мы получаем прирост производительности симуляции благодаря параллелизму. А в виду недетерминированного порядка выполнения этих участков - можем заметить состояния гонки вызванные недостаточно качественной проработкой архитектуры.

\section{Изоляция агентов}

В большинстве случаев агенты должны быть изолированы друг от друга. Это означает, что весь обмен информацией между ними должен происходить посредством сообщений.

В порядке исключения, конечно, можно использовать разделяемую память и мьютексы. Например, это годится для тех случаев, когда мы симулируем распределенное хранилище, а его конкретная реализация и возможность сбоев выходят за рамки эксперимента.

\section{Генерация сценариев}

Генерация сценариев, которые воспроизводятся в симуляции, должна происходить отдельно. Это позволит сравнивать поведение планировщиков и архитектуры в максимально приближенных условиях. Следовательно, симулятор должен предоставлять два инструмента: для генерации сценариев и для запуска симуляции.

Сценарий состоит из трех частей: конфигурация сети, описание датасетов и контейнеров, последовательность запросов в систему.

Сгенерированный сценарий представляет из себя набор YAML-файлов (вышеуказанные части лежат каждая в своем файле). При желании вместо генерации можно составить сценарий вручную или вносить правки в уже сгенерированный.

\section{Сбор статистики}

Нужно предоставить максимальную свободу в обработке данных. Это приводит к тому, что правильно сделать результатом работы симулятора логи, причем в удобном формате. В логи надо писать, что и когда произошло, а с помощью любого внешнего инструмента можно на основе логов выстроить произвольные метрики.

В данной работе в качестве формата логов был выбран JSON в виду хорошего баланса простоты и гибкости.

\section{Детали реализации компонентов системы}

Теперь подробнее рассмотрим принципы реализации конкретных компонентов архитектуры. Всего есть три типа агентов: Core, Control Unit, Worker

\subsection{Core}

Данный агент при инициализации получает список запросов. Каждому запросу из списка соответсвует порядковый номер тика, когда он должен быть послан в систему. Каждый отдельный запрос посылается случайному Control Unit'у.

\subsection{Control Unit}

К каждому Control Unit'у привязана группа Worker'ов. С данными Worker'ами может общаться только этот Control Unit.

В каждый момент времени ровно один Control Unit является лидером.

Control Unit содержит в себе следующий подкомпоненты:

\begin{description}
	\item[scheduler] - сам планировщик. Критически важна возможность использования одной и той же реализации планировщика как для симуляции, так и для реальных условий. Может либо пополнять очереди выполнения произвольных воркеров, либо делегировать запрос лидеру.
	\item[monitor] - компонент, который используется планировщиком, чтобы узнать глобальное состояние системы
	\item[local queue] - содержит очередь задач для подконтрольных воркеров. Компонент Executor из описания архитектуры в виду простоты своего устройства встроен в этот подкомпонент модели Control Unit.
\end{description}

Подкомпоненты работают в отдельных потоках выполнения.

Control Unit может принимать два типа запросов:

\begin{description}
	\item[Запросы на загрузку датасета или запуск контейнера c указанным датасетом] - данные запросы могут приходить от Core или быть делегированы другим Control Unit'ом. Обрабатываются scheduler'ом.
	\item[Запросы на добавление задачи в очередь подконтрольного Worker'а] - данные запросы приходят от других Control Unit'ов. Причина их появления в том, что при своей работе планировщик рассматривает все Worker'ы системы, а не только принадлежащие его Control Unit'у.
\end{description}

\subsection{Worker}

В один момент времени Worker может исполнять только один запрос. Запросы к нему могут приходить только от связанного с ним Control Unit'а.

Worker может принимать три вида запросов:

\begin{description}
	\item[Загрузка датасета] Время, необходимое на это, вычисляется на основе скорости передачи данных, полученной из описания сети.
	\item[Сборка контейнера] - в данный момент считается, что контейнер собирается за один тик. При желании этот параметр можно изменить.
	\item[Выполнение контейнера с заданным датасетом] - датасет и контейнер должны быть уже загружены на Worker. Время, необходимое на данную задачу, вычисляется по упрощенной модели: воркер имеет параметр производительности во флопсах, а контейнер имеет параметр, сколько флопс нужно на каждый мегабайт данных. Разделив одно на другое, получаем искомое время.
\end{description}
% Глава 6 - Описание созданной системы
\chapter{Результаты}

Результатами работы является описание архитектуры (приведенное в тексте диплома) и реализация симулятора доступная на github: \url{https://github.com/ffloyd/evergrid-go}.

Код симулятора продокументирован для облегчения его последующей разработки и модификации. Технические тонкости реализации и более подробное рассмотрение принципов работы отдельных компонентов вынесены за рамки текста диплома и являются частью документации.

Удобная он-лайн версия документации доступна по адресу: \url{https://godoc.org/github.com/ffloyd/evergrid-go}.

Помимо описания работы компонентов, в документации даны рекомендации по путям эффективной модификации и расширения системы.

При реализации описанной в предыдущей главе модели go показал себя с сильной стороны. Модель параллелизма CSP оказалась крайне удобной для реализации многоагентной системы как с точки зрения внутренней архитектуры, так и с точки зрения получившегося API.

Реализованная как часть симулятора многоагентная среда получилась независимой от остального кода и может быть использована как база для написания других симуляторов.

При реализации для симулятора компонентов CoreUnit, Core и Worker было выявлено множество небольших технических аспектов, касающихся синхронизации работы компонентов. Причем эти аспекты будут актуальны и при написании реализаций этих компонентов для реальных условий. Этот факт подтверждает оправданность идеи, что необходимо тестировать не только сам планировщик, но и принципы взаимодействия остальных частей системы.

Эти тонкие моменты выражены в документации, которой сопровожден код и самом коде. В качестве примера приведу один из таких моментов: "если два последовательных запроса в систему оперируют одним датасетом или контейнером, то второй запрос должен быть послан в систему строго после окончания обработки первого Control Unit'ами". Конкретно этот пример связан с тем, что глобальное состояние системы должно полностью отражать изменения, привнесенные обработкой первого запроса. Иначе может оказаться, что мы, например, дважды запланируем загрузку одного и того же датасета на один Worker.

\section{Установка и использование evergrid-go}

Реализованный программный продукт называется evergrid-go и представляет из себя CLI-приложение (Command Line Interface).

Ниже описан наиболее простой сценарий для установки разработанного симулятора и примеры его использования.

Для начала нам надо установить go и настроить его окружение. Инструкции есть на сайте языка программирования go: \url{https://golang.org/doc/install}. Упрощенный способ для установки на Ubuntu: \url{https://github.com/golang/go/wiki/Ubuntu}

Когда окружение настроено скачиваем и устанавлваем актуальную версию evergrid-go следующей командой:

\begin{lstlisting}[language=bash]
go get github.com/ffloyd/evergid-go
\end{lstlisting}

После этого генерируем сценарий работы (со стандартными настройками) в папке simdata:

\begin{lstlisting}[language=bash]
$GOPATH/bin/evergrid-go gendata test simdata
\end{lstlisting}

Поменять параметры генерации можно через опции, список которых можно увидеть с помощью команды:

\begin{lstlisting}[language=bash]
$GOPATH/bin/evergrid-go gendata -h
\end{lstlisting}

Далее можно запустить симуляции для различных типов планировщика:

\begin{lstlisting}[language=bash]
$GOPATH/bin/evergrid-go simulator simdata/test.yaml -s random
$GOPATH/bin/evergrid-go simulator simdata/test.yaml -s naivefast
$GOPATH/bin/evergrid-go simulator simdata/test.yaml -s naivecheap
\end{lstlisting}

\section{Пример работы}

Ниже приведены примеры результатов работы симуляции на одинаковом сценарии для всех трех тривиальных реализаций планировщика.

Показаны только последние строчки логов, которые содержат статистику использования воркеров.

Как будет видно, планировщики дают результаты соответсвующие их целям. Naive Fast имеет наименьшее значение ''total calculating ticks'', Naive Cheap имеет наименьшее значение ''total money spent'', а все три запуска random scheduler оказались существенно хуже по этим параметрам.

Данные результаты представлены в ознакомительных целях, а сами планировщики представляют из себя довольно наивные реализации, которые не предназначены для работы в реальных условиях.

\subsection{Random Scheduler}

При запросе на загрузку датасета выбирается один случайный воркер и датасет загружается на него.

При запросе на выполнение эксперимента - эксперимент запускается на воркере с уже загруженным датасетом.

Так как работа планировщика рандомизирована приведены результаты трех запусков. Остальные два планировщика выдают одинаковый результат при условии одинакового сценария.

\begin{lstlisting}[caption=Random Scheduler: запуск 1]
Total uploading ticks          simulation=big value=269
Total building ticks           simulation=big value=82
Total calculating ticks        simulation=big value=5336
Total money spent              simulation=big value=4158.3458
\end{lstlisting}

\begin{lstlisting}[caption=Random Scheduler: запуск 2]
Total uploading ticks          simulation=big value=269
Total building ticks           simulation=big value=83
Total calculating ticks        simulation=big value=4450
Total money spent              simulation=big value=2670.4092
\end{lstlisting}

\begin{lstlisting}[caption=Random Scheduler: запуск 3]
Total uploading ticks          simulation=big value=269
Total building ticks           simulation=big value=85
Total calculating ticks        simulation=big value=5353
Total money spent              simulation=big value=3992.3984
\end{lstlisting}

\subsection{Naive Fast Scheduler}

При запросе на загрузку датасета выбираются три наиболее производительных воркера с размером очереди меньше пяти, либо просто три наиболее производительных воркера.

При запросе на выполнение эксперимента - среди воркеров с загруженным (или с запланированным для загрузки) датасетом выбирается наиболее быстрый с размером очереди меньше 5-и, либо просто наиболее быстрый из воркеров с минимальной очередью, если это невозможно.

\begin{lstlisting}[caption=Naive Fast Scheduler]
Total uploading ticks          simulation=big value=807
Total building ticks           simulation=big value=55
Total calculating ticks        simulation=big value=1611
Total money spent              simulation=big value=1361.8439
\end{lstlisting}

\subsection{Naive Cheap Scheduler}

При запросе на загрузку датасета выбираются три наиболее дешевых воркера с размером очереди меньше пяти,
либо просто три наиболее дешевых воркера.

Сравнивается цена за одну минуту работы.

При запросе на выполнение эксперимента - среди воркеров с загруженным (или с запланированным для загрузки) датасетом выбирается наиболее дешевый с размером очереди меньше 5-и, либо просто наиболее дешевый из воркеров с минимальной очередью, если это невозможно.


\begin{lstlisting}[caption=Naive Cheap Scheduler]
Total uploading ticks          simulation=big value=807
Total building ticks           simulation=big value=63
Total calculating ticks        simulation=big value=2574
Total money spent              simulation=big value=497.4759
\end{lstlisting}
% Глава 7 - Технические детали созданной системы
\chapter{Технические детали созданной системы}

%\textit{Большое и подробное руководство по всем компонентам системы, описание узких мест и рекомендации к модификации}

% Заключение
\conclusion

\textit{Вкратце напомнить о том, что было в дипломе}

\textit{Написать о перспективах развития}

\textit{Написать о недостатке на рынке решений, связанных с симуляцией процессов и о том, что ни одно из широко известных не использует языки или фреймворки с акторной или CSP моделью параллелизма и конкуретности}

% Список литературы
\bibliography{thesis}
\bibliographystyle{gost705}

% Приложения
\appendix
%\input{app-a}

\end{document}
