\conclusion

В данной работе на основе общего описания системы Evergrid была сформулирована ее архитектура. На основе этой архитектуры была реализована среда ее моделирования на языке программирования go. Соответсвенно результатами работы являются:

\begin{itemize}
	\item описание архитектуры
	\item рекомендации по ее реализации
	\item симулятор этой архитектуры для анализа работы планировщиков
	\item подробная техническая документация этого симулятора посвященная его использованию и модификации
\end{itemize}

Архитектура и документация приведены в тексте диплома, а исходный код симулятора доступен на github.

Сформулированная архитектура призвана облегчить разработку системы в будущем, а написание своего симулятора в противовес использованию готовых решений обосновано совокупностью следующих факторов:

\begin{itemize}
	\item ни одно из популярных решений в своем исходном виде не поддерживает необходимую архитектуру, а их модификация оказывается не менее сложной задачей чем написание своей реализации
	\item малоизвестные решения рискованно использовать
	\item не требуется столь сложный механизм симуляции который используется в популярных решениях
	\item go удобнее для написания и поддержки подобных систем чем C/С++ или Java
	\item отсутствие подобных симуляторов реализованных на go
	\item возможность интероперабельности с реальной системой увеличивает качество симуляции и самого продукта
\end{itemize}

При написании симулятора учитывалось, что его будут расширять и модифицировать, поэтому особое внимание уделялось изолированности и заменяемости его компонент.

%\textit{Вкратце напомнить о том, что было в дипломе}

%\textit{Написать о перспективах развития}

%\textit{Написать о недостатке на рынке решений, связанных с симуляцией процессов и о том, что ни одно из широко известных не использует языки или фреймворки с акторной или CSP моделью параллелизма и конкуретности}