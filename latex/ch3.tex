\chapter{Архитектура Evergrid}

\section{Основные ограничения}

В моем понимании проектирование архитектуры основывается на:

\begin{itemize}
	\item достаточно подробной постановке задачи
	\item выявлении ограничений
	\item использовании актуальных технологий как составных блоков
	\item бритве Оккама (не переусложняй!)
\end{itemize}

Постановка задачи у нас есть, последние два пункта постараемся соблюдать. Осталось понять какие у нас ограничения. Под ограничениями я понимаю технические аспекты, которые напрямую следуют из постановки задачи и ограничивают нас в свободе выбора тех или иных технологий, тех или иных подходов. В этом разделе я приведу те ограничения, которые я считаю неочевидными или заслуживающими отдельного упоминания.

Первое из них - связанное с безопасностью. Мы используем вычислительные ресурсы предоставляемые третьими лицами. И они имеют неограниченный (читай -- root) доступ к ним. Поскольку процесс предоставления этих ресурсов подразумевается достаточно свободным будет правильно представить злоумышленника на месте поставщика мощностей. Какие потенциальные угрозы он может предоставлять?

\begin{itemize}
	\item захват не принадлежащих ему ресурсов
	\item нарушение работы кластера (например, вмешательство в работу планировщика)
	\item нарушение корректной работы предоставленного вычислительного ресурса (фальсификация результатов, намеренное замедление скорости вычислений и т. п.)
\end{itemize}

Первые две угрозы нивелируются достаточно просто: \textit{на арендуемых ресурсах не должен выполняться код связанный с управлением кластером. Только выполнение задач и отправка результатов.}

Третья - наиболее сложная. Но риски можно свести к минимуму если \textit{вообще не запускать на арендуемых ресурсах компоненты системы}. Естественной реализацией этого принципа является удаленное управление по SSH. Тогда злоумышленник будет знать минимум о текущем состоянии системы. Прочие методы борьбы с этой угрозой уже выходят за рамки этой работы.

Следующее ограничение происходит из того, как мы будем выполнять задачи. Очевидно, что нам нужна виртуализация - иначе наш кластер быстро превратят в ботнет или того хуже. А еще нам нужна скорость - следовательно нам нужна легковесная виртуализация. Третий критерий - технология должна быть простой в использовании. В идеале - проведение эксперимента на своей локальной машине не должно отличаться от выполнения его на наших мощностях. Сложив все эти требования воедино мы получим ответ - Docker. И ответ этот великолепен: Docker отлично отвечает требованиям к удобству воспроизводимости и модификации. А его инструментарий органично вписывается в архитектуру. От предоставляемых мощностей нам тогда требуется:

\begin{itemize}
	\item свободное место на диске
	\item доступ по ssh
	\item корректно работающий docker
	\item возможность мониторинга нагрузки (чтобы регистрировать не относящуюся к нашему сервису нагрузку)
\end{itemize}

Даже неограниченный доступ в Интернет не является жестким требованием - мы можем собрать контейнер на наших машинах и готовый образ передать по ssh.

Возникает вопрос о том, как пользователь должен предоставлять контейнер со своим алгоритмом. Если следовать идеям доступности и простоты модификации напрашивается очевидное решение - Github. Наиболее органично будет работать с специально оформленными github репозиториями, которые содержат все необходимое для сборки контейнера.

Если уж рассмотрели то, как загружать реализации алгоритмов (которые мы условно называем процессорами) - то стоит сказать пару слов о загрузке датасетов. Никаких особых ограничений здесь незаметно. Все что нужно - это иметь системе возможность его закачать на свои машины. Путей достижения этого много и в данной работе они не будут рассматриваться (за исключением перемещения датасетов внутри самой системы).

И последнее ограничение о котором я хочу сказать - это CAP-теорема. Мы имеем дело с распределенной в глобальной сети системой, а в таком случае нельзя забывать о CAP-теореме. Нарушение связи между элементами системы не должно приводить к ее некорректной работе.

\section{Предлагаемая архитектура}

\textit{Сигнал - Действие - Оценка - Результат - Смысл}

\textit{Сказать, что сначала покажем общую картину, а потом будем подробнее рассматривать}

\textit{Картика с архитектурой}

\textit{Назначение и смысл компонентов первого слоя}

\textit{Назначение и смысл компонентов второго слоя}

\textit{Назначение и смысл компонентов третьего слоя}


\section{Выбор технологий для реализаций компонентов}

\textit{Затереть про то, что технологии решают}

\textit{Здесь провести анализ на возможные языки для реализации.}

\textit{Выбор технологий для первого слоя}

\textit{Выбор технологий для второго слоя}

\textit{Выбор технологий для третьего слоя}


\section{Аспекты реализации инфраструктуры. CAP-теорема}

\textit{Сформулировать CAP-теорему.}

\textit{Hashicorp consul}

\textit{CAP: первый слой}

\textit{CAP: второй слой}

\textit{CAP: третий слой}