\chapter{Архитектура Evergrid}

\section{Общая формулировка задачи}

\textit{Описание подхода к формулированию архитектуры: постановка задачи + ограничения + современные технологии + разделяй и властвуй = возможные решения, выберем одно}

\textit{О постановке задачи: юзкейсы и группы пользователей}

\textit{Описание групп пользователей: исследователи и поставщики вычислительных мощностей}

\textit{Описание юзкейсов для исследователей}

\textit{Описание юзкейсов для поставщиков (надо нормальное слово придумать)}

\textit{Ограничения связанные с безопасностью}

\textit{Ограничения связанные с доступностью. CAP-теорема. Не углубляться.}


\section{Предлагаемая архитектура}

\textit{Сказать, что сначала покажем общую картину, а потом будем подробнее рассматривать}

\textit{Картика с архитектурой}

\textit{Назначение и смысл компонентов первого слоя}

\textit{Назначение и смысл компонентов второго слоя}

\textit{Назначение и смысл компонентов третьего слоя}


\section{Выбор технологий для реализаций компонентов}

\textit{Затереть про то, что технологии решают}

\textit{Здесь провести анализ на возможные языки для реализации.}

\textit{Выбор технологий для первого слоя}

\textit{Выбор технологий для второго слоя}

\textit{Выбор технологий для третьего слоя}


\section{Аспекты реализации инфраструктуры. CAP-теорема}

\textit{Сформулировать CAP-теорему.}

\textit{Hashicorp consul}

\textit{CAP: первый слой}

\textit{CAP: второй слой}

\textit{CAP: третий слой}