\chapter{Постановка задач}

Пусть список возможностей из предыдущей главы получился весьма лаконичным, его реализация - это огромная работа. Напомню, что данный диплом представляет собой лишь один из самых первых этапов - это имеет прямое влияние на те задачи, которые непосредственно решались, и на то, как они решались.

Краткий список того, над чем велась работа, выглядит так:

\begin{itemize}
	\item Четко сформулировать базовые требования к системе Evergrid
	\item Разработать описание архитектуры, удовлетворяющее этим требованиям
	\item Для данной архитектуры реализовать среду моделирования для исследования эффективности планировщиков выполнения задач и распределения данных
\end{itemize}


Базовые требования были описаны и обоснованы в предыдущей главе.

\section{Проектирование архитектуры}

Наличие описания архитектуры является необходимым условием для создания среды моделирования, которая, в свою очередь, является темой диплома. Мало того, это описание с одной стороны должно быть достаточно подробным, с другой - максимально общим, чтобы не "замораживать" спецификацию тех аспектов системы, детали реализации которых несущественны для данной работы. Тем не менее, даже для таких аспектов будет не лишним дать некоторую обоснованную рекомендацию - она может послужить удачной точкой для принятия финального решения в будущем.

Итого, в плане проектирования архитектуры надо решить следующие задачи:

\begin{itemize}
	\item из каких компонентов состоит система
	\item как эти компоненты распределены по физическим машинам
	\item как компоненты взаимодействуют друг с другом
	\item какие технологии подойдут лучше всего для их реализации
\end{itemize}

\section{Реализация симулятора}

Вторая и основная задача диплома - это реализация симулятора. Предметом симуляции является планировщик - та часть системы, которая управляет размещением данных и выполнением задач на подконтрольных ресурсах. Перед непосредственно реализацией симулятора надо ответить на следующие вопросы:

\begin{itemize}
	\item каким требованиям должен отвечать симулятор
	\item есть ли готовые решения либо решения, которые удобны для модификации
\end{itemize}

Причем важно помнить о том, что какие-то аспекты архитектуры могут изменяться, и нужно будет изменять симулятор в соответствии с ними. Подобные корректировки симулятора не должны быть неоправданно сложными.