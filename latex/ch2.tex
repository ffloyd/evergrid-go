\chapter{Постановка решаемых в дипломе задач}

Пусть список возможностей из предыдущей главы получился весьма лаконичным его реализация - это огромная работа. Напомню, что данный диплом представляет собой лишь один из самых первых этапов - это имеет прямое влияние на те задачи которые непосредственно решались и на то как они решались.

Краткий список того, над чем велась работа выглядит так:

\begin{itemize}
	\item Четко сформулировать базовые требования к системе Evergrid
	\item Разработать описание архитектуры удовлетворяющее этим требованиям
	\item Для данной архитектуры реализовать среду моделирования для исследования эффективности планировщиков выполнения задач и распределения данных
\end{itemize}

Базовые требования были описаны и обоснованны в предыдущей главе.

Наличие описания архитектуры является необходимым условием для создания среды моделирования, которая, в свою очередь, является темой диплома. Мало того, что оно должно быть - это описание с одной стороны должно быть достаточно подробным, с другой - максимально общим, чтобы не "замораживать" спецификацию тех аспектов системы, детали реализации которых несущественны для данной работы. Тем не менее даже для таких аспектов будет не лишним дать некоторую обоснованную рекомендацию - она может послужить удачной точкой для принятия финального решения в будущем. Именно с таким пониманием вопроса велась работа об описании архитектуры.

Касательно описания архитектуры отдельно хочу отметить решения о выборе языков и технологий для реализации. Конечно, все они имеют по большей части рекомендательный характер сопровожденный развернутым комментарием "почему подобный выбор хорош". Единственное исключение - это использование языка go для написания внутренних сервисов приложения и его распределенной части. Причины этого выбора и сравнение с альтернативными решениями будут рассмотрены отдельно и более подробно. Забегая вперед скажу, что среда моделирования написана на go с расчетом на то, что и определенный набор компонентов тоже будет написан на go. Именно из-за этого этому техническому выбору уделяется особое внимание.

Теперь к основному: среда моделирования. Я не буду сейчас вдаваться в подробности реализации и принципы ее работы. По сути, постановка задачи здесь получается достаточно вольной: необходимо иметь возможность тестировать и сравнивать работу различных алгоритмов планировщиков. Конкретные требования к среде моделирования, ее особенности и конкурентные преимущества сформировались в процессе работы над дипломом. Однако есть одно свойство, о котором можно смело говорить до того, как будет описана архитектура Evergrid: полученный результат должен быть хорошо продокументирован и удобен для модификации. Это требование естественным образом происходит из факта раннего этапа разработки самого Evergrid.

Помимо непосредственно реализации среды моделирования нужны две-три несложных реализации алгоритмов планировщика. Причем они должны быть различными по своей сути и приоритетам, выдавать предсказуемый результат. Из-за своей тривиальности они могут послужить хорошей исходной точкой для сравнения - если разработанный боевой алгоритм работает хуже тривиального - значит он не такой уж и "боевой".