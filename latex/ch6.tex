\chapter{Результаты}

Результатами работы является описание архитектуры (приведенное в тексте диплома) и реализация симулятора доступная на github: \url{https://github.com/ffloyd/evergrid-go}.

Код симулятора продокументирован для облегчения его последующей разработки и модификации. Технические тонкости реализации и более подробное рассмотрение принципов работы отдельных компонентов вынесены за рамки текста диплома и являются частью документации.

Помимо описания работы компонентов, в документации даны рекомендации по путям эффективной модификации и расширения системы.

При реализации описанной в предыдущей главе модели go показал себя с сильной стороны. Модель параллелизма CSP оказалась крайне удобной для реализации многоагентной системы как с точки зрения внутренней архитектуры, так и с точки зрения получившегося API.

Реализованная как часть симулятора многоагентная среда получилась независимой от остального кода и может быть использована как база для написания других симуляторов.

При реализации для симулятора компонентов CoreUnit, Core и Worker было выявлено множество небольших технических аспектов, касающихся синхронизации работы компонентов. Причем эти аспекты будут актуальны и при написании реализаций этих компонентов для реальных условий. Этот факт подтверждает оправданность идеи, что необходимо тестировать не только сам планировщик, но и принципы взаимодействия остальных частей системы.

Эти тонкие моменты выражены в документации, которой сопровожден код. В качестве примера приведу один из таких моментов: "если два последовательных запроса в систему оперируют одним датасетом или контейнером, то второй запрос должен быть послан в систему строго после окончания обработки первого Control Unit'ами". Конкретно этот пример связан с тем, что глобальное состояние системы должно полностью отражать изменения, привнесенные обработкой первого запроса. Иначе может оказаться, что мы, например, дважды запланируем загрузку одного и того же датасета на один Worker.