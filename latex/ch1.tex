\chapter{О системе Evergrid}

В первую очередь стоит подробнее объяснить, что такое Evergrid, какие предпосылки обуславливают его актуальность и новизну, какие задачи он призван решать и каковы сценарии его использования. Все это важно еще и тем, что задает основу для разработки и описания подходящей архитектуры, а без нее невозможно говорить о написании симулятора. В конце главы будет пара слов о самом симуляторе, а начнем мы с предпосылок появления Evergrid.

Согласно Эрику Шмидту из Google, теперь каждые два дня человеческая раса создает столько информации, сколько мы производили от начала нашей цивилизации до 2003 года. Это что-то около пяти эксабайт информации в день. Если копнуть глубже, то не совсем понятно, откуда он взял эти цифры\footnote{\url{http://readwrite.com/2011/02/07/are-we-really-creating-as-much/}} , но сложно спорить с основной идеей - информации ежедневно производится беспрецедентно много. Эрик Шмидт озвучил свою мысль в 2010 году. Логично предположить, что спустя 6 лет информационные потоки стали только мощнее. Такое количество производимой информации делает более актуальной проблему ее обработки и экспериментов над ней. Есть даже такие на первый взгляд странные исследования, как определение мест землетрясений в реальном времени с помощью твиттера\cite{Sakaki:2010:EST:1772690.1772777}. Разработка подобных методов зачастую связана с экспериментами над большими массивами данных (гигабайты информации) и, если когда-то работа с такими объемами была мало распространена, то сейчас этим уже никого не удивить. Соответсвенно, инструменты, которые помогают в этих задачах, востребованы на рынке. Как один из многочисленных примеров можно привести Apache Spark.

Помимо инструментов есть еще один не менее важный фактор - коммуникация. Эффективное развитие науки невозможно без эффективной коммуникации между отдельными учеными, университетами, исследовательскими центрами и т. д. Доступность информации - тоже, по сути, является элементом коммуникации. Написание многих современных научных работ было бы более трудоемким без таких инструментов, как Google Scholar, без отлаженной работы научных изданий и прочих факторов, связанных с эффективным обменом результатами исследований. Вывод из этого можно сделать такой: чем лучше коммуникация - тем эффективней развивается наука.

Если говорить об экспериментах над данными, то в плане коммуникации возникает одна очень заметная проблема: воспроизведение экспериментов. Ведь, чтобы воспроизвести эксперимент, надо воссоздать инфраструктуру, в которой он был проведен, и уже на этом этапе порой начинаются проблемы. Не всегда инфраструктура достаточно подробно описана в самой работе. Не всегда с первого раза получается заставить работать описанную инфраструктуру. Не всегда ее можно воспроизвести на локальной машине с используемой исследователем операционной системой. Этот список возможных технических проблем можно продолжить на несколько страниц. И то он будет неполным. И, когда исследователь, желающий повторить оригинальный эксперимент, но с другими данными, сталкивается с такой массой проблем - он рискует потратить много времени на несущественные для его работы вещи.

Также важен открытый доступ к результатам эксперимента. На данный момент с этим нет особых проблем, но в контексте Evergrid стоит об этом упомянуть: результаты должны быть открыты и доступны. Думаю, можно даже не объяснять ценность этого фактора.

Помимо воспроизводимости эксперимента важной является и возможность его модификации. Я говорю о случаях, когда описан сам алгоритм, а его реализация либо не является частью работы, либо сложна в использовании. Evergrid должен учитывать этот фактор и делать модификацию эксперимента максимально доступной. Это означает, что мы должны ввести некоторые стандарты оформления для реализации. На самом деле это вытекает не только из идеологических соображений, но и из технических (сложно сделать систему, которая будет запускать что угодно и как угодно). Более того, введение подобных требований в первую очередь облегчит воспроизводимость. Итого: Evergrid должен задавать некоторый формат (или форматы) оформления программных реализаций ради удобства модификации и воспроизводимости.

Итак, Evergrid, очевидно, нацелен на большое количество пользователей. Это означает, что нужно много вычислительных мощностей для обеспечения адекватного выполнения задач. Часть мощностей можно получить бесплатно, но их почти наверняка не хватит. Соответсвенно, в системе будут присутствовать те мощности, за которые придется платить. И здесь есть два подхода: покупать эти мощности самим или быть посредником. Второй подход интереснее и более чем имеет право на жизнь в виду того факта, что зачастую кластеры различного размера порой просто простаивают. Если мы дадим возможность продавать эти мощности, то выгода будет всем сторонам. Пользователям - т. к. будет ценовое разнообразие, команде Evergrid - т. к. это неплохая бизнес-модель, поставщикам мощностей - получить выгоду от простаивающих кластеров и просто сделать благое дело. Именно реализация этой схемы является одной из ключевых особенностей Evergrid как проекта.

Итак, мы развернуто сформулировали основные идеологические предпосылки для создания Evergrid. Теперь мы можем их выразить в виде лаконичного и короткого списка. Evergrid - это сервис:

\begin{itemize}
	\item для выполнения экспериментов над данными
	\item для публикации результатов этих экспериментов
	\item с возможностью удобно модифицировать исходный эксперимент как в плане алгоритмов, так и в плане данных
	\item с возможностью опубликовать результат выполнения модификации
	\item являющийся посредником между поставщиками вычислительных мощностей и конечными пользователями
	\item задающий некий формат или форматы оформления программной части экспериментов ради удобства воспроизведения и модификации (в том числе и на локальных машинах пользователей)
\end{itemize}

Опираясь на этот список, мы теперь можем сформулировать основные сценарии использования системы. Но полноценные сценарии использования системы в виде диаграмм в данной работе будет делать опрометчиво. Во-первых, это работа для UX-специалиста, коим я сейчас не являюсь. Во-вторых, в рамках данной работы это неактуально. Важно выявить список \textit{возможностей} системы, а архитектуру проектировать исходя из того, чтобы эти возможности в ее рамках эффективно реализовывались. Вот этот список, разбитый по группам пользователей:

\begin{itemize}
	\item Исследователи:
	\begin{itemize}
		\item Возможность загружать датасеты в систему
		\item Возможность загружать процессоры (программные реализации экспериментов) в систему
		\item Возможность запустить определенный процессор на определенном датасете
		\item Возможность задать ограничения на выполнение (уложиться в указанную стоимость, предпочитать быстрое выполнение, несмотря на стоимость, начать выполнение строго до определенного времени и пр.)
		\item Возможность получить результат выполнения
		\item Возможность опубликовать связку датасет+процессор+результат
		\item Возможность "склонировать" опубликованный эксперимент и вносить изменения
	\end{itemize}
	\item Гости:
	\begin{itemize}
		\item Возможность просматривать опубликованные эксперименты
		\item Возможность скачивать результаты, датасеты и процессоры
	\end{itemize}
	\item Поставщики мощностей:
	\begin{itemize}
		\item Возможность видеть статистику использования ресурсов
		\item Возможность модифицировать параметры использования ресурсов
		\item Возможность предоставлять новые ресурсы, закрывать доступ к существующим.
	\end{itemize}
\end{itemize}

Исследователи - это, очевидно, те, кому необходимо проводить эксперименты над данными. Гости - незарегистрированные пользователи. Их функционал доступен всем. Поставщики мощностей - понятно из наименования.

Это относительно грубый список, но достаточный для поставленных задач.

Отдельно упомяну про реализованный симулятор: полноценно говорить о его актуальности, новизне и практической значимости без описания архитектуры не получится, но, если забежать вперед и попробовать сформулировать краткий список, то он будет таким:

\begin{itemize}
	\item нет симуляторов на go, и в то же время go популярен для микровервисов
	\item популярные симуляторы (simgrid, gridsim, alea 2, netlogo) не используют языки с акторной или CSP-моделью
	\item популярные симуляторы не поддерживают необходимую архитектуру без существенных модификаций
	\item модификация существующего решения не менее сложна, чем написание своего
	\item изоморфность (использование одного и того же кода как для симуляции, так и для реального окружения) дает существенные преимущества
	\item симуляция архитектуры на раннем этапе может заблаговременно выявить ее недостатки
	\item расширяемость и модифицируемость созданного симулятора
	\item готовая отправная точка для создания остальных компонентов системы
\end{itemize}
